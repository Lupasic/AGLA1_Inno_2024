\documentclass[a4paper,10pt]{article}
%\documentclass[a4paper,10pt]{scrartcl}

\usepackage[utf8]{inputenc}
\usepackage[russian]{babel} % Cyrillics!
\usepackage{cmap} % Copypastable cyrillics!
\usepackage{amsmath} % Multiline equations
\usepackage{amssymb} % Cool symbols (like |R)
\usepackage{mathtools} % Disable numbering equations by default
\mathtoolsset{showonlyrefs}
\usepackage{indentfirst} % Indent first paragraph in chapter
\usepackage{multicol}


\makeatletter
\renewcommand*\env@matrix[1][*\c@MaxMatrixCols c]{%
  \hskip -\arraycolsep
  \let\@ifnextchar\new@ifnextchar
  \array{#1}}
\makeatother

\newcommand{\T}{\textrm{T}}
\newcommand{\I}{{-1}}

\title{Class \#4}
\author{Oleg Bulichev}
\date{}

\usepackage{color}
\newcommand{\adv}[0]{\textcolor{red}{*}}

\makeatletter
\pdfinfo{%some PDF metadate
  /Title    (\@title)
  /Author   (\@author)
  /Subject  (Analytical Geometry and Linear Algebra I)
}
\makeatother

\begin{document}

\section*{Analytical Geometry and Linear Algebra~I, \\ HW \#4}
\noindent\textbf{Innopolis University, September 2022}
\\

\section{Inverse Matrix}
\begin{enumerate}
\item Find inverse matrices for the following matrices:
\begin{enumerate}
    \item $\begin{bmatrix}2&2&-1\\2&-1&2\\-1&2&2\end{bmatrix}$;
    \item $\begin{bmatrix}1&1&1\\1&2&2\\2&3&4\end{bmatrix}$;
    \item $\begin{bmatrix}1&2&2&2\\2&1&2&2\\2&2&1&2\\2&2&2&1\end{bmatrix}$;
\end{enumerate}

\item Solve matrix equations:
\begin{enumerate}
    \item  $X\begin{bmatrix}2&2&-1\\2&-1&2\\-1&2&2\end{bmatrix}=\begin{bmatrix}5&5&2\\5&8&-1\end{bmatrix}$.
\end{enumerate}

\item It is known that $A^2+A+I=O$ ($O$ is a zero matrix) for a square matrix $A$. Is it true that matrix $A$ is invertible? If it is so, how can we find the inverse matrix?



\end{enumerate}

\section{Matrix Rank}
\begin{enumerate}
\item Calculate the ranks of the following matrices:
\begin{enumerate}

    \item  $\begin{bmatrix}13&16&16\\-5&-7&-6\\-6&-8&-7\end{bmatrix}$;

\end{enumerate}

% \item Determine the rank of $A-\lambda I$ for all values of $\lambda$ if
% \begin{enumerate}

%     \item  $A=\begin{bmatrix}3&0&0\\2&6&4\\-2&-3&-1\end{bmatrix}$;
%     \item  $A=\begin{bmatrix}0&0&-1&0\\0&0&0&1\\1&0&0&0\\0&-1&0&0\end{bmatrix}$.

% \end{enumerate}

\end{enumerate}

\section{Changing Basis and Coordinates}
\begin{enumerate}
\item
There are two different coordinate systems in space: $O$, $\textbf{e}_1$, $\textbf{e}_2$, $\textbf{e}_3$ and $O'$, $\textbf{e}'_1$, $\textbf{e}'_2$, $\textbf{e}'_3$. It is known that the old coordinates $x$, $y$, $z$ are expressed through the new coordinates $x'$, $y'$, $z'$ with the following formulas: $$x=x'+y'+z'-1;\;y=-x'+z'+3;\;z=-x'-y'-2.$$

(a) Find the transition matrix from the new basis to the old one and the transition matrix from the old basis to the new one.

(b) Find the coordinates of $O$, $\textbf{e}_1$, $\textbf{e}_2$, $\textbf{e}_3$ in the new coordinate system.

(c) Find the coordinates of $O'$, $\textbf{e}'_1$, $\textbf{e}'_2$, $\textbf{e}'_3$ in the old coordinate system.
\item
Let us consider two coordinate systems in the plane: $O$, $\textbf{e}_1$, $\textbf{e}_2$ and $O'$, $\textbf{e}'_1$, $\textbf{e}'_2$. Point $O'$ has coordinates $(7;-2)$ in the old coordinate system, and vectors $\textbf{e}'_1$, $\textbf{e}'_2$ can be obtained from vectors $\textbf{e}_1$, $\textbf{e}_2$ by rotating them $60^{\circ}$ (a) clockwise; (b) counterclockwise. Find the old coordinates of a point $x$, $y$ given its new coordinates $x'$, $y'$.


\end{enumerate}

\end{document}