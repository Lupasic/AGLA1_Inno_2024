% \documentclass[aspectratio=169,notes]{beamer}
\documentclass[aspectratio=169]{beamer}
\usetheme[faculty=phil]{fibeamer}
\usepackage{polyglossia}
\setmainlanguage{english} %% main locale instead of `english`, you
%% can typeset the presentation in either Czech or Slovak,
%% respectively.
\setotherlanguages{russian} %% The additional keys allow
%%
%%   \begin{otherlanguage}{czech}   ... \end{otherlanguage}
%%   \begin{otherlanguage}{slovak}  ... \end{otherlanguage}
%%
%% These macros specify information about the presentation
\title[AGLA1]{Analytical Geometry and Linear Algebra I, Lab 11} %% that will be typeset on the
\subtitle{Affine Transformation
\\ \   \\ \ 
         } %% title page.
\author{Oleg Bulichev}
%% These additional packages are used within the document:
\usepackage{ragged2e}  % `\justifying` text
\usepackage{booktabs}  % Tables
\usepackage{tabularx}
\usepackage{tikz}      % Diagrams
\usetikzlibrary{calc, shapes, backgrounds}
\usepackage{amsmath, amssymb}
\usepackage{url}       % `\url`s
\usepackage{listings}  % Code listings
% \usepackage{subfigure}
\usepackage{floatrow}
\usepackage{subcaption}
\usepackage{mathtools}
\usepackage{todonotes}
\usepackage{fontspec}
\usepackage{multicol}
\usepackage{pdfpages}
\usepackage{wrapfig}
\usepackage{animate}
\usepackage{booktabs}
\usepackage{multirow}

\graphicspath{{resources/}}
\frenchspacing

\setbeamertemplate{caption}[numbered]
\usetikzlibrary{graphs}

% \usepackage[backend=biber,style=ieee,autocite=footnote]{biblatex}
% \addbibresource{biblio.bib}
% \DefineBibliographyStrings{english}{%
%   bibliography = {References},}

\newcommand{\oleg}[2][] {\todo[color=red, #1] {OLEG:\\ #2}}
\newcommand{\fbckg}[1]{\usebackgroundtemplate{\includegraphics[width=\paperwidth]{#1}}}%frame background

\usepackage[framemethod=TikZ]{mdframed}
\newcommand{\dbox}[1]{
\begin{mdframed}[roundcorner=3pt, backgroundcolor=yellow, linewidth=0]
\vspace{1mm}
{#1}
\vspace{1mm}
\end{mdframed}
}

\begin{document}
\setlength{\abovedisplayskip}{0pt}
\setlength{\belowdisplayskip}{0pt}
\setlength{\abovedisplayshortskip}{0pt}
\setlength{\belowdisplayshortskip}{0pt}

\fbckg{fibeamer/figs/title_page.png}
\frame[c]{\setcounter{framenumber}{0}
    \usebeamerfont{title}%
    \usebeamercolor[fg]{title}%
    \begin{minipage}[b][6.5\baselineskip][b]{\textwidth}%
        \textcolor{black}{\raggedright\inserttitle}
    \end{minipage}
    % \vskip-1.5\baselineskip

    \usebeamerfont{subtitle}%
    \usebeamercolor[fg]{framesubtitle}%
    \begin{minipage}[b][3\baselineskip][b]{\textwidth}
        \raggedright%
        \insertsubtitle%
    \end{minipage}
    \vskip.25\baselineskip
}
%   \frame[c]{\maketitle}

\fbckg{fibeamer/figs/common.png}

% \begin{frame}[t]{Questions for today}
%     \framesubtitle{}
%     \begin{itemize}
%         \item How can I work with general form of 2nd order curve equation?
%         \item How it relates with cone?
%         \item What forms of equation do we have?
%     \end{itemize}
% \end{frame}



% \begin{frame}[t]{Reference material}
%     % \framesubtitle{OnlineMschool}
%     \Large
%     \begin{itemize}
%         \item \href{https://ppt-online.org/842240}{Conic sections (slides, rus)}
%         \item \href{https://www.youtube.com/watch?v=GaSrBIUGdWs}{How to find centre of conic (video, eng)}
%         \item \href{https://math.stackexchange.com/questions/2235355/find-major-and-minor-axis-of-given-conic}{Find the equation of major and minor axis of the given conic}
%         \item \href{http://mathprofi.ru/kak_privesti_uravnenie_linii_2_poryadka_k_kanonicheskomu_vidu.html}{How to go from general to canonical form (mathprofi, rus)}
%     \end{itemize}
% \end{frame}

\fbckg{fibeamer/figs/last_page.png}
\frame[plain]{}

\end{document}